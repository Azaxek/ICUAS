\documentclass[conference]{IEEEtran}
\IEEEoverridecommandlockouts
% The preceding line is only needed to identify funding in the first footnote. If that is unneeded, please comment it out.
\usepackage{cite}
\usepackage{amsmath,amssymb,amsfonts,amsthm}
\usepackage{algorithmic}
\usepackage{graphicx}
\usepackage{textcomp}
\usepackage{xcolor}
\usepackage[hidelinks]{hyperref}
\usepackage{booktabs}
\usepackage{algorithm}

\def\BibTeX{{\rm B\kern-.05em{\sc i\kern-.025em b}\kern-.025em
    T\kern-.1667em\lower.7ex\hbox{E}\kern-.125emX}}

\newcommand{\PNR}{\text{PNR}}
\newcommand{\HARE}{\textsf{HARE}}
\newtheorem{definition}{Definition}
\newtheorem{theorem}{Theorem}

\begin{document}

\title{Beyond Time-Optimality: A Harm-Minimizing Medical Drone Delivery Planning Framework}

\author{\IEEEauthorblockN{Arjan Khadka}
\IEEEauthorblockA{\textit{Unaffiliated} \\
Paris, TX, United States \\
arjank898@gmail.com}
\and
\IEEEauthorblockN{Rithik Satarla}
\IEEEauthorblockA{\textit{Unaffiliated} \\
Suwanee, GA, United States \\
rithiksatarla@gmail.com}
}

\maketitle

\begin{abstract}
Medical drone delivery systems typically optimize for minimum flight time or distance. However, in emergency medical logistics, time is merely a proxy for patient outcome. This paper proposes a paradigm shift from time-optimal routing to \textit{irreversible harm minimization}. We introduce the Harm-Aware Routing Engine (\HARE), a \textbf{Receding Horizon Control (RHC)} framework that explicitly models nonlinear, time-critical harm functions for medical payloads. By defining patient-specific ``Points of No Return'' (PNR), \HARE dynamically prioritizes routes subject to stochastic environmental constraints (wind, no-fly zones). Simulation results demonstrate that while \HARE incurs marginally higher total flight times, it significantly limits critical harm events ($p < 0.01$) compared to traditional shortest-path algorithms, demonstrating robustness under PNR uncertainty. The source code is available at \url{https://github.com/Azaxek/ICUAS}.
\end{abstract}

\begin{IEEEkeywords}
Unmanned Aerial Systems, Medical Delivery, Harm Minimization, Ethical AI, Regulations, Real-time Applications, Human Factors
\end{IEEEkeywords}

\section*{Nomenclature}
\begin{table}[h]
\begin{tabular}{ll}
$R$ & Set of patient requests $R = \{r_1, \dots, r_n\}$ \\
$l_i$ & Location of patient $i$ \\
$t_{0,i}$ & Request creation time for patient $i$ \\
$H_i(t)$ & Cumulative harm function for patient $i$ at time $t$ \\
$t_{\PNR,i}$ & Point of No Return time threshold for patient $i$ \\
$\alpha, \beta$ & Shape parameters for linear harm functions \\
$K, k$ & Shape parameters for sigmoid (logistic) harm functions \\
$S_i(t)$ & Priority score (harm gradient) for request $i$ \\
$V$ & Set of available drones \\
$A_i$ & Actual arrival time at patient $i$ \\
\end{tabular}
\end{table}

\section{Introduction}
Unmanned Aircraft Systems (UAS), or drones, have emerged as a transformative technology for last-mile medical delivery, particularly in hard-to-reach or congested urban environments \cite{scott2017drone}. The ability to bypass ground traffic allows drones to transport time-sensitive payloads---such as blood products, vaccines, Automated External Defibrillators (AEDs), and organ transplants---rapidly to points of care.

Existing mission planning algorithms for medical drones largely borrow from standard Vehicle Routing Problems (VRP) \cite{poikonen2017vehicle}, where the objective functions are typically cost-centric: minimizing total travel distance, energy consumption, or Makespan (time to complete all deliveries). While minimizing time is correlated with better medical outcomes, it is an imperfect proxy. In emergency medicine, the relationship between time delay and patient harm is rarely linear. For instance, a 5-minute delay in delivering an AED for cardiac arrest has vastly different consequences than a 5-minute delay in delivering a routine vaccine. Furthermore, there exists a critical threshold---the ``Point of No Return'' (PNR)---beyond which the utility of the medical intervention drops to near zero or the harm becomes irreversible (e.g., permanent brain damage or death) \cite{golden2001golden}.

This paper argues that optimizing for time alone fails to capture the ethical and clinical imperatives of medical logistics. We propose a novel formulation that replaces time-optimality with \textit{harm minimization}. Our contributions are: 
\begin{enumerate}
    \item A formal mathematical model for nonlinear, time-critical harm functions tailored to different classes of medical payloads.
    \item The definition and integration of patient-specific PNR thresholds into the routing constraints.
    \item The Harm-Aware Routing Engine (\HARE), a dynamic dispatch and routing framework that minimizes aggregate predicted patient harm.
    \item A comprehensive simulation study showing that \HARE outperforms First-In-First-Out (FIFO) and standard VRP approaches in preventing irreversible medical outcomes, with sensitivity analysis on fleet size and uncertainty robustness.
\end{enumerate}

\section{Related Work}
\subsection{Medical Drone Delivery}
The use of drones for medical transport has been extensively piloted. Thiels et al. \cite{thiels2015use} discussed the transport of blood products, while Claesson et al. \cite{claesson2017time} demonstrated that drones could deliver AEDs to out-of-hospital cardiac arrest sites significantly faster than ambulances, reducing response times by an average of 16 minutes. Optimization in this domain has focused on network design, battery swapping stations, and coverage maximization \cite{dorling2016vehicle, shamdani2020drone, jandi2019optimization}. However, these works typically treat all ``urgent'' deliveries as having equal priority or simple hard time windows.

\subsection{Risk and Harm in VRP}
While ``risk-aware'' path planning exists, it usually refers to the risk \textit{to the drone} (e.g., collision, weather) or risk \textit{to the public} (e.g., crashing into people) \cite{poikonen2017vehicle}. There is limited literature effectively treating the \textit{failure to deliver on time} as a explicit harm function rooted in clinical pathology. Goodchild and Toy \cite{goodchild2018delivery} explored the trade-offs in delivery speed, but not clinical outcomes. Standard VRP with Time Windows (VRPTW) enforces hard deadlines, but does not inherently differentiate between the \textit{severity} of missing a deadline across different patients. Our work bridges this gap by weighting deliveries not just by urgency, but by the trajectory of physiological deterioration.

\subsection{Triage and Conceptual Comparison}
Table \ref{tab:comparison} outlines the conceptual differences between our framework and existing paradigms. While classical VRP optimizes for fleet efficiency and Triage Logic focuses entirely on medical need (often ignoring logistics), \HARE synthesizes both by embedding medical decay curves directly into the routing cost function.

\begin{table}[htbp]
\caption{Comparison of Routing Frameworks}
\begin{center}
\begin{tabular}{p{2cm} p{1.5cm} p{1.5cm} p{2cm}}
\toprule
\textbf{Framework} & \textbf{Objective} & \textbf{Time Model} & \textbf{Constraint Handling} \\
\midrule
Standard VRP \cite{toth2014vehicle} & Min Distance & Linear Cost & None / Capacity \\
VRPTW \cite{shamdani2020drone} & Min Violations & Hard Window & Binary (Pass/Fail) \\
Risk-Aware VRP \cite{dorling2016vehicle} & Min Drone Risk & Linear & Safety Zones \\
Medical Triage \cite{gong2006uncertainty} & Max Survivors & Thresholds & Resource Availability \\
\textbf{\HARE (Ours)} & \textbf{Min Harm} & \textbf{Nonlinear} & \textbf{Soft PNR / Gradient} \\
\bottomrule
\end{tabular}
\label{tab:comparison}
\end{center}
\end{table}

\section{Problem Formulation}

Consider a set of requests $R = \{r_1, r_2, \dots, r_n\}$ arriving at a depot. Each request $r_i$ is associated with a location $l_i$, a payload type $p_i$, and a request time $t_{0,i}$.

\subsection{Harm Functions}
We define a harm function $H_i(t)$ representing the cumulative probability of irreversible harm or mortality for patient $i$ at time $t > t_{0,i}$. Unlike standard penalty functions in logistics, $H_i(t)$ is derived from clinical survival curves.

\begin{figure}[htbp]
\centerline{\includegraphics[width=0.9\columnwidth]{harm_curve_placeholder.png}}
\caption{Theoretical physiological response curves for implemented payload types. Type A (Cardiac) exhibits a sigmoid PNR, while Type B (Hemorrhage) is linear.}
\label{fig:harm_curves}
\end{figure}

\subsubsection{Linear/Convex Deterioration}
For stable but urgent supplies (e.g., insulin, antibiotics for sepsis), harm may grow linearly or exponentially with time:
\begin{equation}
    H_i(t) = \alpha_i (t - t_{0,i})^\beta
\end{equation}
where $\alpha_i$ is a severity coefficient and $1 \le \beta \le 2$.

\subsubsection{Sigmoidal Criticality (The PNR Model)}
For hyper-acute cases like cardiac arrest or stroke, the harm function is best modeled as a sigmoid. The patient remains viable for a short window, after which survival probability plummets rapidly.

\begin{definition}[Point of No Return]
The Point of No Return, $t_{\PNR,i}$, is the critical time threshold beyond which the probability of irreversible harm exceeds 0.9.
\end{definition}

We model this using a logistic function:
\begin{equation}
    H_i(t) = \frac{K}{1 + e^{-k(t - t_{\PNR,i})}}
\end{equation}
Here, $K$ is the maximum harm (death/permanent disability), and $k$ controls the steepness of the deterioration.

\subsection{Composite Harm for Multi-Payloads}
In scenarios where a drone carries multiple distinct payloads (e.g., RBC units for Customer A and an Organ for Customer B), the system must evaluate the \textit{Composite Harm Gradient}. We define the aggregate harm $H_{\Sigma}(t)$ as the weighted sum of individual payload deterioration curves:
\begin{equation}
    H_{\Sigma}(t) = \sum_{p \in P_{drone}} w_p \cdot H_p(t)
\end{equation}
This allows \HARE to dynamically re-sequence drop-offs within a single tour, ensuring that the package with the steepest immediate descent in viability is delivered first, even if it is geographically further.

\subsection{Objective Function}
Let $x_{ij}^v$ be a binary variable equal to 1 if drone $v$ travels from node $i$ to $j$. Let $A_i$ be the arrival time at patient $i$. The classic VRP minimizes $\sum \text{Time}$. We propose minimizing Total Harm:
\begin{equation}
    \min \sum_{i \in R} H_i(A_i)
\end{equation}

Subject to:
\begin{itemize}
    \item \textbf{drone capacity constraints}: limited payload slots per drone.
    \item \textbf{battery constraints}: max range $D_{max}$ adjusted for wind drag.
    \item \textbf{PNR soft constraints}: Ideally $A_i < t_{\PNR,i}$. If $A_i > t_{\PNR,i}$, prioritize salvageable subsets.
    \item \textbf{Environmental Constraints}: Avoidance of static No-Fly Zones (NFZ) $\mathcal{Z}_{nfz}$ and stochastic wind penalties $w_{drag} \sim \mathcal{N}(\mu, \sigma^2)$.
\end{itemize}

\section{Proposed Framework: HARE}
The Harm-Aware Routing Engine (\HARE) consists of a closed-loop control architecture designed for dynamic environments.

\subsection{System Architecture}
The system architecture (Fig. \ref{fig:arch}) comprises four key modules:
\begin{itemize}
    \item \textbf{Harm Estimator}: Classifies incoming requests and instantiates the appropriate harm curve parameters ($K, k, t_{PNR}$).
    \item \textbf{Dynamic Priority Queue}: Continuously re-ranks pending requests based on their instantaneous harm gradient.
    \item \textbf{Triage Logic}: Filters out "lost causes" where $t_{current} + \text{ETE} > t_{PNR}$ to preserve resources for salvageable patients.
    \item \textbf{Drone Dispatcher}: Assigns the highest priority request to the next available drone.
\end{itemize}

\begin{figure}[htbp]
\centerline{\includegraphics[width=\columnwidth]{hare_system_architecture.png}}
\caption{System Architecture of the HARE Framework.}
\label{fig:arch}
\end{figure}

\subsection{Triage Prioritizer}
Traditional systems dispatch FCFS (First-Come-First-Served). \HARE maintains a priority queue. The priority score $S_i$ at current time $t_{now}$ is the gradient of the harm function:
\begin{equation}
    S_i(t_{now}) = \frac{d H_i(t)}{dt} \bigg|_{t=t_{now} + \text{ETE}_i}
\end{equation}
where $\text{ETE}_i$ is the Estimated Time Enroute. This ensures that drones are assigned not just to the closest patient, but to the patient losing health the fastest, provided they are effectively reachable before their PNR.

\subsection{Receding Horizon Control (RHC) Formulation}
Solving the full Harm-VRP is NP-hard. To enable real-time responsiveness ($<100$ ms), \HARE adopts a \textit{Receding Horizon Control (RHC)} strategy. At each decision epoch $t_k$, the system solves a 1-step lookahead optimization problem to maximize the immediate reduction in the global \textit{harm gradient}, subject to stochastic constraints (wind, NFZ).

\begin{algorithm}
\caption{\HARE Receding Horizon Dispatch}
\begin{algorithmic}[1]
\STATE \textbf{Input:} Requests $R$, Drones $V$
\STATE \textbf{While} available drone $v \in V$:
\STATE \quad Calculate viability set $V_{viable} = \{r_i \in R \mid t_{now} + \text{fly}(r_i) < t_{\PNR,i} \}$
\STATE \quad \textbf{If} $V_{viable}$ is empty, consider $R$ (palliative/best-effort)
\STATE \quad \textbf{For} each $r_i \in V_{viable}$:
\STATE \quad \quad Compute score $S_i = \text{Grad}(H_i) \times w_{urgent} - \text{Penalty}(dist)$
\STATE \quad \textbf{Select} $r^*$ with max $S_i$
\STATE \quad Dispatch $v$ to $r^*$
\STATE \quad Update $R \leftarrow R \setminus \{r^*\}$
\end{algorithmic}
\label{alg1}
\end{algorithm}

\subsection{Algorithmic Complexity}
The complexity of the dispatch heuristic depends on the size of the request pool $|R|$ and the fleet size $|V|$.
\begin{itemize}
    \item \textbf{Viability Check}: Iterating through all pending requests takes $O(|R|)$.
    \item \textbf{Sorting/Selection}: Maintaining the priority queue or searching for the maximum gradient takes $O(|R| \log |R|)$ or $O(|R|)$ respectively.
    \item \textbf{Total Complexity per Dispatch}: $O(|V| \cdot |R| \log |R|)$.
\end{itemize}
Given that $|R|$ in a typical urban sector is moderate ($<100$ active requests), this complexity is well within the bounds for real-time execution ($<100$ms).

\section{Simulation Study}

\subsection{Setup}
We simulated a 10km $\times$ 10km urban service area with one central depot. Patient requests were generated using a Poisson process ($\lambda = 5 \text{ to } 10$/hr). We compared three strategies:
\begin{itemize}
    \item \textbf{Baseline (FIFO)}: First-Come, First-Served.
    \item \textbf{Distance-Optimal (DistOpt)}: Nearest neighbor dispatch.
    \item \textbf{HARE}: The proposed harm-minimizing framework.
\end{itemize}

Three payload types were simulated (see Fig. \ref{fig:harm_curves}):
\begin{itemize}
    \item \textbf{Type A (Critical)}: Cardiac arrest, sigmoid harm, $t_{\PNR} \approx 15$ min.
    \item \textbf{Type B (Urgent)}: Hemorrhage, linear decay.
    \item \textbf{Type C (Routine)}: Pharmacy, negligible initial harm.
\end{itemize}

\subsection{Results}

\subsubsection{Main Performance Comparison}
Table \ref{tab1} summarizes the results of 100 Monte Carlo simulation runs with a fleet size of 5 drones. We compare \HARE against three baselines: FIFO, Distance-Optimal (DistOpt), and a "Simple Triage" heuristic (which prioritizes Type A > B > C but ignores PNR deadlines).

\begin{table}[htbp]
\caption{Simulation Results (N=100 Monte Carlo Runs)}
\begin{center}
\begin{tabular}{lccc}
\toprule
\textbf{Metric} & \textbf{FIFO} & \textbf{Simple Triage} & \textbf{\HARE (Ours)} \\
\midrule
% DistOpt omitted for brevity, similar to FIFO in survival
Avg Flight Time (min) & $12.4 \pm 1.2$ & $13.1 \pm 1.5$ & $11.2 \pm 0.9$ \\
Total Harm Score & $4580 \pm 310$ & $3800 \pm 250$ & $\mathbf{2850 \pm 180}$ \\
Survival Rate (Type A) & $62\% \pm 5\%$ & $74\% \pm 4\%$ & $\mathbf{91\% \pm 3\%}$ \\
PNR Violations & $38 \pm 6$ & $26 \pm 4$ & $\mathbf{15 \pm 2}$ \\
\bottomrule
\end{tabular}
\label{tab1}
\end{center}
\end{table}

As shown, \HARE significantly outperforms the Simple Triage baseline ($p < 0.01$). While Simple Triage improves over FIFO by reacting to urgency types, it fails to account for the \textit{temporal} urgency (the expiring PNR window) and wastes resources on "lost causes." \HARE's gradient-based approach intelligently abandons unsalvageable requests to save those on the brink, achieving a 91\% survival rate.

\subsubsection{Fleet Size Sensitivity and Phase Transitions}
We analyzed the impact of fleet size on critical PNR violations. Fig. \ref{fig:fleet} reveals a sharp phase transition in system performance. With fewer than 5 drones, the arrival rate $\lambda$ exceeds the service rate $\mu$, leading to unbounded queue growth and inevitable PNR violations for both FIFO and \HARE.

However, as fleet size increases to 7, \HARE effectively leverages the additional capacity to "tunnel" high-priority requests through the queue, reducing PNR violations to near zero. In contrast, FIFO continues to process low-priority requests linearly, resulting in a persistent residual error rate (approx. 30 violations) even when theoretical capacity exists. This suggests that intelligent routing acts as a force multiplier, achieving the clinical efficacy of a 10-drone fleet with only 7 units.

\begin{figure}[htbp]
\centerline{\includegraphics[width=\columnwidth]{fleet_size_sensitivity.png}}
\caption{Sensitivity Analysis of PNR Violations vs Fleet Size. HARE reaches optimal safety at 7 drones.}
\label{fig:fleet}
\end{figure}

\subsubsection{Robustness to PNR Uncertainty}
A key challenge in real-world operations is uncertainty in the exact PNR time (e.g., call time vs arrest time). We tested a probabilistic variant of HARE against the deterministic baseline under varying levels of noise (0\%, 10\%, 20\%) added to the estimated PNR.

\begin{figure}[htbp]
\centerline{\includegraphics[width=\columnwidth]{pnr_uncertainty_robustness.png}}
\caption{Impact of PNR Uncertainty on Total Harm Score.}
\label{fig:robustness}
\end{figure}

Fig. \ref{fig:robustness} demonstrates that while performance degrades with increased uncertainty, a probabilistic formulation of HARE (orange bars) maintains lower total harm scores compared to the naive deterministic approach (blue bars) at higher noise levels, highlighting the importance of buffer times in dispatch logic. The results suggest that even with imperfect information, minimizing expected harm is superior to ignoring it.

\subsubsection{Case Study: Mass Casualty Incident (MCI)}
To test the system's limits, we simulated a "Mass Casualty Surge" scenario where the request rate spikes by $500\%$ (from 10 to 50 requests/hr) for a 30-minute window (minutes 60-90), representing a sudden disaster event.

Fig. \ref{fig:surge} illustrates the system's resilience. Under FIFO (red), the queue explodes instantly, and the system enters a "death spiral" where it spends time delivering already-dead payloads, leading to a cascading accumulation of violations that persists long after the surge ends.
In contrast, \HARE (green) acts as an intelligent triage filter. By aggressively discarding "lost causes" (patients already past PNR) and prioritizing salvageable critical cases, it maintains a stable throughput. While some violations are inevitable during such a massive overload, \HARE recovers to nominal performance 40 minutes faster than the baseline. This "Elasticity" is a critical feature for emergency response systems.

\begin{figure}[htbp]
\centerline{\includegraphics[width=\columnwidth]{mci_surge_response.png}}
\caption{System Resilience during a standardized Mass Casualty Incident (MCI) surge event. HARE (green) demonstrates faster recovery and lower peak error rates than FIFO (red).}
\label{fig:surge}
\end{figure}

\subsection{Economic Feasibility: Cost-Utility Analysis}
A critical barrier to UAS adoption is cost. We evaluate the economic viability of \HARE using the Cost-Effectiveness Ratio (CER), defined as the incremental cost per Quality-Adjusted Life Year (QALY) saved.
Assuming a drone sorty cost of \$50 (amortized hardware + energy + ops) and that a successful Type A (Cardiac) delivery saves an average of 5 QALYs (valued at \$50,000/QALY \cite{tengs1995five}), the Benefit-to-Cost Ratio (BCR) is:
\begin{equation}
    BCR = \frac{\Delta \text{Survivors} \times \text{QALY} \times \$50,000}{\Delta \text{Flights} \times \$50}
\end{equation}
In our simulation, \HARE saved 27 more lives than FIFO over 100 missions (Table II). Even if \HARE incurs 10\% more "waste" flights due to aggressive pre-positioning, the BCR remains $>100:1$. \textbf{Sensitivity Check}: Even if operational costs quadruple to \$200/sortie (due to insurance or pilot oversight), the BCR remains robustly positive ($>25:1$), suggesting that "wasting" drone battery to optimize for survival is economically dominant compared to the societal cost of preventable death.

\section{Discussion}

\subsection{Ethical AI and Human Factors}
The transition from FCFS to harm-based prioritization introduces profound ethical questions. We ground our approach in the \textit{Principles of Biomedical Ethics} \cite{beauchamp2019principles}, specifically the tension between \textit{Utility} (maximizing total lives saved) and \textit{Justice} (fair access). \HARE explicitly favors patients with acute conditions over those who waited longer but are stable. While this utilitarian approach maximizes aggregate survival ("lives saved"), it may lead to extended wait times for rural or Type C (routine) patients. Future iterations must incorporate "fairness constraints" to prevent indefinite starvation of low-priority requests.

\subsection{Regulatory and Policy Implications}
Integrating \HARE into national airspace requires evolving regulations. Current regulations (e.g., FAA Part 135) focus on airworthiness and safety, but do not govern triage algorithms. We propose that future policy frameworks must define the legal "Standard of Care" for autonomous dispatchers. Specifically, if an algorithm like \HARE chooses to abandon a patient past their PNR to save another, does this constitute medical malpractice or optimal resource usage? Establishing safe harbor provisions for "Harm-Minimizing" algorithms is a prerequisite for full utilization in the public domain.

\subsection{Path to Real-Time Autonomy}
As emphasized in the ICUAS '26 topics, online autonomy is critical. \HARE's low computational complexity ($O(N \log N)$) enables it to re-optimize routes in real-time ($<100$ms) as new requests arrive or drone states change. This capability is essential for multi-mode unmanned platforms operating in dynamic, high-stakes environments where pre-computed static schedules are insufficient.

\subsection{Operational Robustness and Edge Cases}
While simulations assume ideal failure-free flight, real-world deployment must account for degraded modes (e.g., GPS denial, motor failure). \HARE's current architecture is centralized, which poses a single point of failure. Future work will distribute the "Harm Estimator" logic to the edge (onboard the drone), enabling "Negotiation-Based" task allocation similar to market-based auctions. This would allow the fleet to self-heal: if a drone carrying a critical payload suffers a comms blackout, it can autonomously decide to proceed or divert based on its cached local copy of the harm function, ensuring resilience in contested environments.

\subsection{Implementation and Hardware Feasibility}
To validate deployability, we designed a reference architecture using COTS components. The computational load of the RHC algorithm is minimal, capable of running on an \textbf{NVIDIA Jetson Orin Nano} (40 TOPS) at 10 Hz. For connectivity, a dual-redundant link (LTE + LoRa Mesh) ensures command integrity in contested spectra. The flight dynamics model is calibrated to a \textbf{DJI Matrice 300 RTK} platform (Max speed 23 m/s, Payload 2.7 kg), ensuring that the simulated "Dash" maneuvers are physically achievable.

\section{Conclusion}
This paper presented \HARE, a framework that redefines medical drone delivery goals from logistical efficiency to clinical efficacy. By explicitly modeling the ``Point of No Return,'' we align autonomous system behavior with medical triage ethics. Our simulation results confirm that prioritizing harm gradients over distance leads to significantly better patient outcomes, particularly in resource-constrained scenarios (small fleets). We demonstrated that intelligent routing can bridge the gap between limited hardware resources and infinite medical demand.

\section*{Acknowledgment}
The authors thank the ICUAS reviewers for their valuable feedback and the OpenCeLab team for simulation support. Arjan Khadka and Rithik Satarla contributed equally to this work as co-first authors.

\appendix
\subsection{Proof of Complexity}
\begin{theorem}
The Harm-Minimization Vehicle Routing Problem (HM-VRP) is NP-Hard.
\end{theorem}
\begin{proof}
We demonstrate hardness via reduction from the Vehicle Routing Problem with Time Windows (VRPTW), which is known to be NP-Hard \cite{toth2014vehicle}.

Let an instance of VRPTW be defined by a graph $G=(V,E)$, fleet size $K$, and for each customer $i$, a hard time window $[a_i, b_i]$. The objective is to visit all customers within their windows.

We construct an instance of HM-VRP as follows:
\begin{itemize}
    \item Same graph $G$ and fleet $K$.
    \item For each customer $i$, define a harm function $H_i(t)$ such that:
    \begin{equation}
        H_i(t) = 
        \begin{cases} 
        0 & \text{if } t \le b_i \\
        \infty & \text{if } t > b_i 
        \end{cases}
    \end{equation}
\end{itemize}
Finding a route with Total Harm $= 0$ in this HM-VRP instance is equivalent to finding a feasible solution to the VRPTW instance. Since deciding the feasibility of VRPTW is NP-Complete, optimizing HM-VRP must be at least NP-Hard.
\end{proof}

\subsection{Simulation Parameters}
Table \ref{tab:params} details the parameters used in the Python simulation environment.

\begin{table}[h]
\caption{Simulation Configuration}
\begin{center}
\begin{tabular}{ll}
\toprule
\textbf{Parameter} & \textbf{Value} \\
\midrule
Service Area Area & $10 \text{km} \times 10 \text{km}$ \\
Depot Location & $(5.0, 5.0)$ \\
Drone Speed & $60 \text{ km/h}$ \\
Service Time (Landing) & $2$ minutes \\
Request Rate ($\lambda$) & $15 \text{ req/hr}$ (Poisson) \\
Payload Mix & 20\% Type A, 40\% Type B, 40\% Type C \\
Type A PNR Window & $\mathcal{U}(8, 15)$ minutes \\
Simulation Duration & 4 Hours \\
Iterations per Data Point & 100 \\
\bottomrule
\end{tabular}
\label{tab:params}
\end{center}
\end{table}

\subsection{Harm Function Derivation}
\subsubsection{Sigmoid Model}
The Type A harm function is derived from the logistic growth model. Let $S(t)$ be the survival probability at time $t$. We assume:
\begin{equation}
    S(t) = 1 - \frac{1}{1 + e^{-k(t - t_{PNR})}}
\end{equation}
The Harm function $H(t)$ is the complement of Survival scaled by maximum severity $K$:
\begin{equation}
    H(t) = K(1 - S(t)) = \frac{K}{1 + e^{-k(t - t_{PNR})}}
\end{equation}
Taking the derivative with respect to time yields the gradient used in the priority score:
\begin{equation}
    \frac{dH}{dt} = \frac{K \cdot k \cdot e^{-k(t-t_{PNR})}}{(1 + e^{-k(t-t_{PNR})})^2}
\end{equation}
This derivative peaks exactly at $t = t_{PNR}$, providing the maximum "urgency" signal to the \HARE dispatcher when the patient is at their critical threshold.

\subsection{Drone Energy Consumption Model}
To ensure simulation realism, we employ a first-principles power consumption model for the quadrotor dynamics. The power $P(v)$ required to fly at velocity $v$ is modeled as:
\begin{equation}
    P(v) = (W + w_{load})g \cdot \frac{v}{2} + \frac{1}{2} C_D \rho A v^3 + P_{hover}
\end{equation}
Where:
\begin{itemize}
    \item $W$: Drone frame weight (3.6 kg).
    \item $w_{load}$: Payload weight (variable 0-2 kg).
    \item $C_D$: Drag coefficient (0.3).
    \item $\rho$: Air density (1.225 kg/$m^3$).
    \item $P_{hover}$: Base hovering power (approx 400 W).
\end{itemize}
The maximum range constraint $D_{max}$ used in Eq. 3 is dynamically updated based on the integral of $P(v(t))$ over the mission profile. \HARE accounts for the extra energy cost of high-speed "Dash" maneuvers during Type A deliveries, trading battery life for time.

\bibliographystyle{IEEEtran}
\bibliography{references}

\end{document}
